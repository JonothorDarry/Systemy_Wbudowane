\documentclass[12pt]{article}
\usepackage[polish]{babel}
\usepackage[MeX]{polski}
\usepackage[utf8]{inputenc}
\usepackage[T1]{fontenc}
\usepackage{lmodern}
\usepackage{graphicx}
\usepackage{subcaption}
\usepackage{listings}
\usepackage[svgnames]{xcolor}
\usepackage[a4paper,bindingoffset=0.2in,%
left=1in,right=1in,top=1.5in,bottom=1.5in,%
footskip=.25in]{geometry}
\pagenumbering{gobble}
\graphicspath{{./zdjs/}}


\begin{document}
\title{Sprawozdanie - Odtwarzanie muzyki z nut na pięciolinii}
\author{Marcin Zatorski 136834, Sebastian Michoń 136770}
\date{\vspace{-2ex}}
\maketitle

\section{Cel i zakres projektu}
Celem projektu było zaimplementowanie aplikacji odczytującej nuty z obrazu z kamery, a następnie odtworzenie ich za pomocą głośnika z użyciem płytki Raspberry Pi.

\begin{figure}[h!]
	\centering
	\includegraphics[width=0.9\linewidth]{SW-schematic.png}
	\caption{Schemat połączeń}
	\label{fig:schemat}
\end{figure}
	
\section{Projekt a realizacja}
Aplikacja została napisana w języku Python i Cython. Rozpoznawanie nut zostało zaimplementowane przy użyciu biblioteki OpenCV. Aplikacja jest w stanie rozpoznać nuty z wysoką skutecznością, o ile zdjęcie zostało zrobione w dobrych warunkach oświetleniowych. Algorytm rozpoznaje tylko część symboli - nuty (całe nuty, półnuty, ćwierćnuty i ósemki) oraz klucze. Tą część algorytmu można by rozszerzyć o rozpoznawanie większej ilości symboli. Dokładność rozpoznawania nut również można by ulepszyć na przykład poprzez zastosowanie maszynowego uczenia się.
	
W trakcie rozwijania projektu dużą przeszkodą była szybkość działania. Dzięki przepisaniu części kodu do języka Cython oraz optymalizacjom aplikacja działa zadowalająco szybko.
	
Pierwotnie w projekcie zakładaliśmy użycie płytki BeagleBone Black. Zdecydowaliśmy się jednak na użycie płytki Raspberry Pi - umożliwiło to proste podłączenie głośnika przez Bluetooth.
	
Aplikacja nie pozwala na wybranie dźwięku instrumentu, co było początkowo planowane. Aplikacja nie odtwarza nut z obu pięciolinii jednocześnie - są odtwarzane po kolei.
	
Aplikację można rozwinąć o lepszy interfejs użytkownika. Obecnie aplikacja jest uruchamiana z linii poleceń. Warto by było dodać na przykład GUI pokazujące obraz z kamery lub przycisk umożliwiający wykonanie zdjęcia. Brak interfejsu utrudnia ocenę, czy aplikacja działa poprawnie.

\section{Kluczowe fragmenty kodu}

\section{Zdjęcia fizycznych połączeń urządzeń}
\begin{figure}[h!]
	\centering
	\begin{subfigure}[b]{0.32\linewidth}
		\includegraphics[width=\linewidth]{Zdj0.png}
		\caption{Pierwszy podpis}
	\end{subfigure}
	\begin{subfigure}[b]{0.32\linewidth}
		\includegraphics[width=\linewidth]{Zdj0.png}
		\caption{Drugi podpis}
	\end{subfigure}
	\begin{subfigure}[b]{0.32\linewidth}
		\includegraphics[width=\linewidth]{Zdj0.png}
		\caption{Trzeci podpis}
	\end{subfigure}
	
	\begin{subfigure}[b]{0.48\linewidth}
		\includegraphics[width=\linewidth]{Zdj0.png}
		\caption{Czwarty podpis}
	\end{subfigure}
	\begin{subfigure}[b]{0.48\linewidth}
		\includegraphics[width=\linewidth]{Zdj0.png}
		\caption{Piąty podpis}
	\end{subfigure}
\end{figure}

\section{Podsumowanie, wnioski}
Aplikacja realizuje swój cel, choć są elementy które można poprawić lub które nie zostały zaimplementowane. Użyty przez nas algorytm uzyskuje wysoką skuteczność przy dobrej jakości zdjęcia; zasadna była by próba ulepszenia kodu poprzez użycie metod adaptatywnych, opartych na funkcji kosztu zamiast metod analitycznych przetwarzania obrazu.

\end{document}
